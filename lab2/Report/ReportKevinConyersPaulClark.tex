\documentclass[12pt, titlepage]{article}
\usepackage{amsthm}
\usepackage{amsthm}
\usepackage{amsmath}
\usepackage{setspace}
\usepackage{lipsum}
\usepackage{graphicx}
\graphicspath{{./images/}}
\usepackage{titling}
\usepackage{indentfirst}
\usepackage{amssymb}
\usepackage{mathtools}
\usepackage{amsfonts}
\doublespacing{}
\usepackage[backend=biber, style=authoryear]{biblatex}
\title{Lab Two Report}
\author{Paul Clark \\ Kevin Conyers}
\begin{document}
\maketitle{}
\section*{Part One}
\par{} 
The objective of this lab is to explore the Data Shared problem between two processes.
To do this we were provided with code that instantiates a parent process and a child process.
These two processes share a variable called countptr. 
Theses two processes are supposed to increment the variable by different values,
the parent process is supposed to increment by twenty each time it runs, while the child
process is supposed to increment by two each time it runs.
So if the parent process is called twice, and the child is called 3 times, the final value is should be forty-six.
\subsection*{Question One}
\par{} 
To test if this is the case, we compile the code in lab2-1.c and run it. 
If the code is correct we should see that countptr ends up being equal to
the number of times parent process is called multiplied by twenty plus the number 
of times child process runs multiplied by two.
The following is the output from a run of the given code:
\begin{center}
	Child process --\textgreater\textgreater counter= 2\\
Child process --\textgreater\textgreater counter= 4\\
Child process --\textgreater\textgreater counter= 6\\
Child process --\textgreater\textgreater counter= 8\\
Child process --\textgreater\textgreater counter= 10\\
Child process --\textgreater\textgreater counter= 12\\
Child process --\textgreater\textgreater counter= 14\\
Child process --\textgreater\textgreater counter= 16\\
Child process --\textgreater\textgreater counter= 18\\
Parent process --\textgreater\textgreater counter = 20\\
Child process --\textgreater\textgreater counter= 20\\
Child process --\textgreater\textgreater counter= 22\\
Child process --\textgreater\textgreater counter= 24\\
Child process --\textgreater\textgreater counter= 26\\
Child process --\textgreater\textgreater counter= 28\\
Child process --\textgreater\textgreater counter= 30\\
Child process --\textgreater\textgreater counter= 32\\
Child process --\textgreater\textgreater counter= 34\\
Child process --\textgreater\textgreater counter= 36\\
Child process --\textgreater\textgreater counter= 38\\
Child process --\textgreater\textgreater counter= 40\\
Parent process --\textgreater\textgreater counter = 40\\
Child process --\textgreater\textgreater counter= 42\\
Child process --\textgreater\textgreater counter= 44\\
Child process --\textgreater\textgreater counter= 46\\
Child process --\textgreater\textgreater counter= 48\\
Child process --\textgreater\textgreater counter= 50\\
Parent process --\textgreater\textgreater counter = 60\\
\end{center}
\par{}
So the child process is called twenty-five times and the parent process is called three times.
However, counter's final value is sixty, instead of 110 as would be expected.
So clearly this code is not working as intended.
\subsection*{Question Two}
\par{}
The variable countptr is most certainly not a shared variable.
The changes made by Child Process are not visible by the parent, or vice versa.
\section*{Part Two}
\par{}
In part two we were tasked with modifying lab2-2.c so that Peterson's solution
was properly implemented.
\subsection*{Question One}
\par{}
This is the run long from the unedited lab2-2.c
\begin{center}
Child process --\textgreater\textgreate counter= 5\\
Child process --\textgreater\textgreater counter= 9\\
Child process --\textgreater\textgreater counter= 13\\
Child process --\textgreater\textgreater counter= 17\\
Child process --\textgreater\textgreater counter= 21\\
Child process --\textgreater\textgreater counter= 25\\
Child process --\textgreater\textgreater counter= 29\\
Child process --\textgreater\textgreater counter= 33\\
Child process --\textgreater\textgreater counter= 37\\
Parent process --\textgreater\textgreater counter = 40\\
Child process --\textgreater\textgreater counter= 41\\
Child process --\textgreater\textgreater counter= 45\\
Child process --\textgreater\textgreater counter= 49\\
Child process --\textgreater\textgreater counter= 53\\
Parent process --\textgreater\textgreater counter = 66\\
Child process --\textgreater\textgreater counter= 5\\
Child process --\textgreater\textgreater counter= 9\\
Child process --\textgreater\textgreater counter= 13\\
Child process --\textgreater\textgreater counter= 17\\
Child process --\textgreater\textgreater counter= 21\\
Child process --\textgreater\textgreater counter= 25\\
Child process --\textgreater\textgreater counter= 29\\
Child process --\textgreater\textgreater counter= 33\\
Child process --\textgreater\textgreater counter= 36\\
Child process --\textgreater\textgreater counter= 40\\
Parent process --\textgreater\textgreater counter = 41\\
Child process --\textgreater\textgreater counter= 44\\
Child process --\textgreater\textgreater counter= 48\\
Child process --\textgreater\textgreater counter= 53\\
Parent process --\textgreater\textgreater counter = 66\\
Child process --\textgreater\textgreater counter= 5\\
Child process --\textgreater\textgreater counter= 9\\
Child process --\textgreater\textgreater counter= 13\\
Child process --\textgreater\textgreater counter= 17\\
Child process --\textgreater\textgreater counter= 21\\
Child process --\textgreater\textgreater counter= 25\\
Child process --\textgreater\textgreater counter= 29\\
Child process --\textgreater\textgreater counter= 33\\
Child process --\textgreater\textgreater counter= 37\\
Parent process --\textgreater\textgreater counter = 40\\
Child process --\textgreater\textgreater counter= 41\\
Child process --\textgreater\textgreater counter= 45\\
Child process --\textgreater\textgreater counter= 49\\
Child process --\textgreater\textgreater counter= 53\\
Parent process --\textgreater\textgreater counter = 66\\
\end{center}
\par{}
THe counter is vible to both the child an parent process.
\subsection*{Question Two}
\par{}
Clearly this is not functioning as intended, as the counter
is not increasing by on two's and twenty's
Instead the counter is increasing unpredictably because 
access to it is not controlled.
\end{document}
